%\chapter{Aufbau der Arbeit}

Jede Arbeit besteht in der Regel aus einer \textbf{Problemstellung}, einem \textbf{definitorischen Abschnitt}, der eigentlichen \textbf{Behandlung der Problemstellung} sowie einer \textbf{Zusammenfassung der zentralen Ergebnisse}.

\begin{description}

	\item[Einleitung] Im Zentrum des erstens Teils stehen die Darstellung des Themas der Arbeit und die genaue Auflistung der Fragestellungen (Wieso ist das Thema relevant?). Ebenso sollten schon einzelne Aspekte des Problems herausgearbeitet werden. Dabei ist es hilfreich, die zentralen Fragen aufzulisten, die im Rahmen der Arbeit beantwortet werden sollen.
	
	Außerdem sollte ein knapper Überblick gegeben werden, in welchen Schritten die Problembehandlung erfolgt: Hinführung zum Thema, Herleitung und Ausformulierung der Fragestellung, Abgrenzung des Themas (Angabe von Aspekten, die zum Thema gehören, aber ausgeklammert werden) und Aufbau der Arbeit (Begründung der Gliederung).
	
	\item[Grundlagen (definitorischer Teil)] Im zweiten Teil sollen zentrale Begriffe definiert und eingeordnet werden. Es geht dabei nicht darum, Definitionen aus Lexika zu suchen; stattdessen sollten problemorientierte Definitionen verwendet werden. Häufig können einzelne Begriffe unterschiedlich weit oder eng definiert werden, sodass auch eine Diskussion unterschiedlicher Definitionsansätze hilfreich sein kann, bevor eine für die weitere Arbeit verbindliche Definition gewählt wird. Zudem sollte ein Überblick über die in der Literatur vorhandenen Methoden bzw. Lösungsansätze, der aktuelle Stand der Technik und verwandte Arbeiten gegeben werden.
	
	\item[Hauptteil] Im Hauptteil der Arbeit (der in der Gliederung selbstverständlich nicht so zu benennen ist\ldots) erfolgt die eigentliche eigentliche Auseinandersetzung mit der Problemstellung. In diesem Teil kommt es darauf an, nicht nur Lehrbuchwissen zusammenzutragen, sondern die Problemstellung reflektiert zu bearbeiteten. Aussagen sollten durch herangezogene Literatur gestützt und belegt werden. Bitte darauf achten, in logischen, nachvollziehbaren Schritten vorzugehen.
	
	\item[Schlussbetrachtung] Die Antwort auf die in der Problemstellung aufgeworfenen Fragen soll kurz und prägnant zusammengefasst werden. Ebenso sollte ein Ausblick auf offen gebliebene Fragen sowie auf interessante Fragestellungen, die sich aus der Arbeit ergeben, gegeben werden. Eine kritische Betrachtung der eigenen Arbeit ist an dieser Stelle ebenfalls sinnvoll.

\end{description}

\noindent
Eine Sammlung unserer Tipps für das Schreiben von Ausarbeitungen befindet sich online unter \url{https://www.dcl.hpi.uni-potsdam.de/media/theses/}.
