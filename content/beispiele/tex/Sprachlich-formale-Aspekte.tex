%\chapter{Sprachlich-formale Aspekte}

Wissenschaftliche Ausarbeitungen dienen der Wissensvermittlung -- es ist überaus wichtig, Lesenden die Informationsaufnahme möglichst einfach zu machen, Inhalte logisch zu gliedern und in guter sprachlicher Form darzustellen.


\section{Textverständlichkeit}

Der Text ist logisch aufzubauen und so zu formulieren, dass er auch für den nicht an der Durchführung der Arbeit beteiligten Lesenden verständlich und nachvollziehbar ist. Die behandelten Themen müssen leicht erkennbar sein. Größere Abschnitte sollten einen kurzen Überblick über ihre Inhalte geben. Die Verständlichkeit des Textes kann durch die Verwendung von kurzen Sätzen, einer einfachen, aber fachsprachlich korrekten Wortwahl und durch die Vermeidung von Füllworten und überflüssigen Fremdworten wesentlich erhöht werden.

\begin{itemize}
\item Keine Prosa, sondern präzise Begriffe und Sätze!
\item Eine einheitliche Terminologie verwenden, damit Begriffe wiedererkannt werden können.
\item Zentrale Begriffe klären und die Arbeit für interessierte Laien verständlich halten.
\item Pure Textblöcke, die sich über mehrere Seiten erstrecken, sind ein Zeichen für mangelnde strukturelle Leseunterstützung. Weiter unterteilen oder variantenreichere Inhaltsarten (Schriftarten, Listen, Diagramme, Tabellen, \ldots) einsetzen.
\item Das schnelle überfliegen des Textes und das Springen in der Arbeit muss aktiv unterstützt werden. Lesende müssen jederzeit die wesentlichen gerade diskutierten Themen schnell erkennen und auch bestimmte vorher schon einmal gelesene Fakten schnell wiederfinden können.
\item Kurz einen Gesamtüberblick (Einordnung ins >>große Bild<<, eine Tabelle) geben und dann tiefer in die Details. Z. B. vor dem Start einer Reihe von Subsections die verschiedene Ausprägungen eines Sachverhalts diskutieren, diese Sachverhalte vorher alle aufzählen und kurz erläutern.
\item Fremdworte/Fachbegriffe nicht einfach ohne weitere Erläuterung verwenden und als bekannt voraussetzen. Selbst wenn der Begriff etabliert und bekannt scheint -- das ist oft auch nur in einem Teilgebiet (der Informatik) so. Deshalb generell Fachbegriffe und Fremdworte erläutern.
\item Die einzelnen Abschnitte sollten entsprechend auf einander verweisen. Überleitungen und Zusammenfassungen zwischen Kapiteln sind hilfreich.
\item Zu Beginn eines Kapitels ist eine Übersicht über dessen Inhalt sinnvoll. Am Ende eines Kapitels kann eine Überleitung zum nächsten Kapitel helfen, den roten Faden aufzuzeigen.
\item Gute Überschriften, vielseitige Präsentation der Inhalte (Diagramme, Tabellen, Auflistungen, \ldots) und aussagekräftige Inhaltsunterschriften verwenden.
\item Durch eine Kombination aus Text und Bild lassen sich komplexe Sachverhalte vereinfacht darstellen und verständlich vermitteln.
\item Verwendet sprechende Titel für Kapitel/Sections/\ldots! Nicht einfach nur >>Aufbau<<, >>Mechanismen<<, >>Dritter Schritt<<, etc. Man sollte nicht erst den Text lesen müssen, um den Kontext zu verstehen. Viel besser: >>Aufbau einer Ausführungsumgebung für Microservices<<, >>Mechanismen zur Fehlervermeidung und Fehlerbeseitigung<<, >>Dritter Schritt: Implementierung der Schnittstellen zwischen Diensten<<.
\item Statt in der Textform z. B. >>mittels einerseits \ldots andererseits<<,\\>>erstens\ldots zweitens\ldots drittens<< o. ä. lieber Aufzählungszeichen verwenden. Dies unterstützt die Lesbarkeit teilweise enorm.
\item Möglichkeiten zur Hervorhebung (z. B. Fettdruck) und Textstrukturierung (Gedankenstriche, Klammern, Semikolon, Doppelpunkt, \ldots) nutzen.
\end{itemize}



\section{Ausdruck \& Stil}

Eine wissenschaftliche Ausdrucksweise ist sachlich, präzise und bemüht sich um Objektivität. Die Verwendung umgangssprachlicher Ausdrücke, schwammiger Formulierungen und übertriebener literarischer Stilmittel (z. B. Verwendung von Synonymen) stören die wissenschaftliche Ausdrucksweise.

\begin{itemize}
\item Umgangssprachliche Formulierungen vermeiden (>>von vorneherein<<, >>wird es richtig teuer<<, >>ziemlich simpel<<, >>Gehen wir das ganze einmal durch<<, >>zum Laufen zu bringen<<, >>sprich\ldots<<, >>Fazit: \ldots<<, >>Ich habe mir gedacht,\ldots<<)
\item Komponenten nicht personifizieren (>>der JBoss/er<<, >>die Apaches<<, >>JBosse<<).
\item    Vermeidet das Wort >>offensichtlich<<. Das wirkt, als hieltet ihr die Lesenden für dumm.
\item    Vermeidet Füllwörter wie >>sehr<<. Wenn etwas >>sehr wichtig<< ist, dann sind in der Schriftsprache Worte wie >>zentral<<, >>fundamental<<, >>essentiell<<, etc. eleganter.
\item    Worte wie >>sehr<<, >>relativ<<, >>ziemlich<<, >>quasi<<, >>gewissermaßen<< sind in den allermeisten Fällen überflüssig und ungenau.
\item    Die Begriffe, für die Demonstrativpronomen (dieser/jener/welcher) Stellvertreter sind, müssen eindeutig erkennbar sein.
\item    Nicht zu umständliche Stellvertreterausdrücke verwenden (>>der zur Diskussion stehende Sachverhalt<<, >>die vorbezeichneten Gegenstände<<, etc.) -- da müssen Lesende viel zu viel nachdenken (und erstmal den Lesefluss stoppen und nachgucken, welche drei Sachen eigentlich gemeint sind).
\item    Nicht verschiedenste Synonyme für ein und denselben Begriff verwenden -- insbesondere, wenn der Begriff etabliert ist (Negativbeispiel: >>künstliche neuronale Netze<<, >>artifizielle Netze<<, >>die in Rede stehenden Netze<<, >>ebenjene Netze<<, >>die beschriebenen Netze<<)
\end{itemize}



\section{Rechtschreibung \& Grammatik}

Mindestens ebenso wichtig wie die Verständlichkeit ist die sprachliche Korrektheit. Ausarbeitungen müssen hinsichtlich Rechtschreibung, Grammatik, Satzbau und Zeichensetzung ohne Fehler sein. Ein nennenswerter Fehleranteil wird oft als Indikator für mangelnde Sorgfalt und Ernsthaftigkeit der Arbeit gewertet. Solche Arbeiten werden nicht anerkannt auch nicht als Vor-Version!).

\begin{itemize}
\item Alles, was die Rechtschreibkontrolle nicht kennt, ist entweder falsch geschrieben oder muss als Fremdwort, Eigenname, \verb|\code| etc. hervorgehoben sein.
\item Es empfiehlt sich generell, Freunde, Kommilitonen, und eine Software zur Prüfung der Rechtschreibung \& Grammatik nochmal auf den Text schauen zu lassen. Als Autor bekommt man schnell einen Tunnelblick und sieht die Fehler nicht mehr.
\item Beachtet schwierige Wörter. >>zum einen<<, >>zum anderen<<, >>des Weiteren<<
\item Regeln für das Setzen von Bindestrichen: Deutsch > immer (>>Hasso-Plattner-Institut<<), Englisch > in der Regel nicht (>>Hasso Plattner Institute<<), Deutsch+Englisch > kombiniert (>>Java EE-Sicherheitsmodell<<). Im Deutschen kommt es äußerst selten vor, dass Worte weder Bindestrich haben noch zusammengeschrieben werden können. Es heißt z. B. nicht >>Download Modus<<, sondern >>Download-Modus<< oder (da Download im Duden steht) >>Downloadmodus<<.
\item Zu einem >>einerseits<< muss es ein >>andererseits<< geben, zu einem >>erstens<< auch ein >>zweitens<<, zu einem >>sowohl<< auch ein >>als auch<<, usw.
\end{itemize}

Schreibung von Zahlen (deutsch):

\begin{itemize}
\item Zahlen von eins bis zwölf werden in der Regel ausgeschrieben. Ansonsten nur ein- und zweisilbige Zahlwörter (hundert, tausend, \ldots)
\item Vor Zeichen, Abkürzungen von Maßen, Gewichten, Geldsorten usw. ist die Zahl in Ziffern zu schreiben: 3 km; 7,4 kg; 6 EUR. Steht statt der Abkürzung die entsprechende Vollform, kann man sowohl in Ziffern als auch in Buchstaben schreiben: 11 Kilometer/elf Kilometer; 2 Euro/zwei Euro.
\item Die Zahlen von 13 an können -- sofern sie Übersichtlich sind -- auch ausgeschrieben werden.
\item Im IT-Bereich gibt es sehr oft einen Unterschied zwischen 0 (dem Zahlenwert) und null (dem Nullwert/NIL, Fehlen eines Wertes)!
\item Zahlen sollten zur besseren Lesbarkeit in Dreiergruppen gegliedert werden, und zwar sowohl links als auch rechts des Dezimaltrennzeichens. Laut ISO 80000 soll das Tausendertrennzeichen ein schmales Leerzeichen sein, niemals ein Komma, Punkt oder irgendein anderes Zeichen. Zahlen sollten (außer bei tabellarischer Darstellung) erst ab fünf Stellen untergliedert werden.
\item Sätze nie mit Konjunktionen (>>und<<, >>oder<<, >>aber<<, sondern) beginnen. Konjunktionen sind -- wie der Name schon sagt -- Verbindungswörter und stellen die syntaktische Verbindung zwischen Wörtern, Satzteilen oder Sätzen her.
\end{itemize}


\section{Grafiken, Tabellen \& Codeausschnitte}

Jede Fließumgebung (Grafiken, Graphen, Diagramme, Tabellen, Codeausschnitte, \ldots) muss beschriftet sein (>>captions<<).

\begin{itemize}
\item Die Beschreibungen der Abbildungen, Tabellen, Diagramme, Quellcode, etc. müssen jene auch ohne Kontext beschreiben -- erläutern, was man alles sehen und erkennen kann. So muss man beim Betrachten nicht zurück in den Fließtext springen.
\item In jeder Beschriftung müssen folgende Fragen beantwortet werden: Was ist dargestellt? Welche Besonderheiten sind zu erkennen? Welche Rückschlüsse ergeben sich daraus für den momentan behandelten Sachverhalt?
\item Die Achsen von Diagrammen ordentlich beschriften (Metrik und Einheiten). Bei Vergleichen angeben, ob große oder ob kleine Werte besser sind. Fehlerbalken verwenden.
\item Wenn Text in den Abbildungen auf Englisch ist, man aber einen deutschen Text schreibt: Entweder den Text in der Abbildungen übersetzen, oder die Bildunterschrift so gestalten, dass man das auch verstehen kann, wenn man kein Englisch kann.
\item Bilder so skalieren, dass die Textgrößen verschiedener Abbildungen etwa konsistent sind und nicht sehr viel größer (oder kleiner) als die normale Textschriftgröße.
\item Grafiken, Tabellen etc. müssen auch im Text referenziert werden um die Verbindung zwischen Text und Abbildungen herzustellen.
\end{itemize}

\section{weitere wichtige Formalien}

Bei den Formalien gibt es verschiedene Möglichkeiten -- die Grundregel sollte jedoch sein: Hauptsache einheitlich, Übersichtlich und systematisch!

\begin{itemize}
\item Fachbegriffe, Produkt-/Eigennamen und fremdsprachlichen Begriffen (z. B. Java EE, Java Virtual Machine, Enterprise Services, Application Client Container) bei der ersten Verwendung kenntlich machen (z. B. mittels \verb|\emph|) und auch eine kurze Erläuterung mit hinzufügen. Oft lässt sich das Erläutern eines Fremdworts/Fachbegriffs leicht durch eine Übersetzung implementieren; manchmal aber auch nicht: Dann muss ein Nebensatz, eine Fußnote o. ä. investiert werden, um den Fachbegriff/das Fremdwort genauer zu erklären. Danach kann auch das Fremdwort normal verwendet werden.
\item Bei englischen Begriffen, die leicht durch deutsche Begriffe ersetzt werden können, dies bitte auch tun (z. B. >>Interface<< vs. >>Schnittstelle<<, >>Button<< vs. >>Schaltfläche<<).
\item Abkürzungen sind grundsätzlich bei ihrer ersten Verwendung einmal aufzuschlüsseln. Auch Begriffe, die im Glossar erwähnt sind, sind bei der ersten Verwendung in der Arbeit noch einmal kurz zu erläutern (z. B. Pan- und Pitch-Gesten).
\item Bei Verwendung von Unterpunkten müssen mindestens zwei Unterpunkte vorhanden sein (also >>2.<< >>2.1<< >>2.2<< \ldots >>3.<< statt >>2<< >>2.1<< >>3<<).
\item Eine Section, auf die sofort eine Subsection (ohne Text dazwischen) folgt, ist unschön.
\item Bei Nennung von Produkten die URL der Bezugsquelle als Fußnote angeben.
\item URLs nicht in den Fließtext integrieren, sondern als Fußnote oder ggf. als Referenz darauf verweisen (sonst unterbrechen sie durch ihre Länge den Lesefluss). Bei Verwendung von LaTeX die URLs immer auch in die \verb|\url|-Umgebung einfügen.
\item Schreiben in der ersten Person Singular vermeiden. >>Ich<< ist üblicherweise nur akzeptabel, wenn es um eigene Leistungen/Beiträge geht.
\item Aufzählung einzelner Begriffe nur machen, wenn sie in der Auflistung noch etwas genauer erklärt werden. Ansonsten einfach als Fließtext hintereinander aufschreiben.
\item Sind Subsections wirklich immer nötig? Oder tut es vielleicht auch eine einfache Auflistung?
\item Keine zusätzlichen Formatierungen in Überschriften verwenden.
\end{itemize}



\section{spezielle Hinweise für Ausarbeitungen, die mit LaTeX bearbeitet werden}

\begin{itemize}
\item Absätze nicht mit \verb|\\| trennen, sondern durch eine Leerzeile. Die beiden Sachen sehen im erstellten Dokument unterschiedlich aus (sonst werden z. B. die Absätze nicht eingerückt).
\item Für Gedankenstriche bitte \verb|--| benutzen (doppeltes Minus).
    Mithilfe einer \verb|~| (Tilde) kann ein geschütztes Leerzeichen (engl. no-break space) eingefügt werden, dass einen automatischen Zeilenumbruch an dieser Position verhindert (bzw. verzögert) und dadurch die Lesbarkeit verbessert (z. B. \verb|123~kg|, \verb|3~Liter|, \verb|DB~Systel|, \verb|S.~42~ff.| oder auch zur Umbruchsteuerung bei Titel-Angaben).
\item Bei allen Quellen, die im Quellenverzeichnis auftauchen sollen, muss irgendwo eine Referenz darauf existieren (kein \verb|\NoCite|!).
\item URLs bitte in die \verb|\url|-Umgebung einfügen, möglichst in eine Fußnote (\verb|\footnote|) packen, den Seitentitel nennen und ggf. das Abrufdatum angeben.
\item Kein \verb|\emph| o. ä. in Überschriften verwenden.
\item Literaturverzeichnis: Als *.bib-Datei!
\item Bei Firmen-/Organisationsbezeichnungen im author-Feld die sich aus mehreren Wörtern zusammensetzen (und die keine Vornamen/Nachnamen sind) diese separat in \verb|{}| packen. Zum Beispiel \verb|{{Microsoft Corporation}}| (damit sie nicht als Vor-/Nachname formatiert werden).
\item Bei mehreren Autoren diese nicht mit Komma voneinander trennen, sondern mit and.
\end{itemize}



\section{Korrektur \& Abgabe}

Eine gute schriftliche Ausarbeitung braucht eine gute Argumentation und eine gute Schlussüberarbeitung. Diese sind jedoch nicht nach ersten Niederschrift fertig -- daher sind mehrere Überarbeitungen vor Abgabe der Endfassung unbedingt notwendig. Kurze und präzise Formulierungen entwickelt man nicht beim ersten Nachdenken über ein Problem. Logikfehler oder fehlende Argumente fallen nicht sofort auf.

\begin{itemize}
\item Den Text mehrfach lesen und überarbeiten.
\item Wiederholungen beseitigen, Abschnitte eventuell umstellen, umformulieren, Brüche glätten, Teile verbinden, Aussagen präzisieren, an der Sprache feilen.
\item Den roten Faden durchgängig kenntlich machen, die Fragestellung und Argumentation schärfen, deren Nachvollziehbarkeit überprüfen.
\item Möglichst auch noch einmal eine (externe) Rechtschreibkontrolle zu Rate ziehen -- eine korrekte Rechtschreibung und Grammatik sind ein Muss!
\item Ebenfalls solltet ihr vor der Abgabe eines Dokuments noch einmal (gründlich) nachschauen, ob alles so aussieht, wie es soll passt das Layout, wurden alle (gravierenden) overfull-Boxes beseitigt, sind die Referenzen ordentlich gesetzt, ist das Literaturverzeichnis vorhanden usw.
\item Wichtig ist auch, dass ihr euch jeden einzelnen Eintrag im Literaturverzeichnis noch einmal anschaut -- sind notwendigen Angaben alle dargestellt, ist die Autorenliste korrekt, sieht man bei Online-Quellen auch die Adresse etc.
\end{itemize}

\section{Drucken \& Binden}

Nach dem Schreiben der Abschluss-Arbeit muss diese noch gedruckt und gebunden werden. Damit das möglichst hochwertig, schnell und preiswert geschehen kann, solltet ihr folgende Dinge beachten:

\begin{itemize}
\item \emph{Papierstärke} Für den Ausdruck bitte ordentliches Papier verwenden (so, dass man die Rückseite nicht durchschimmern sieht). Um professionell zu wirken, sollte Papier mit mind. $100~g/m^2$ gewählt werden.
\item \emph{Bindung} Wir raten dazu, beim Binden ein >>Softcover mit Aufdruck auf der Vorderseite<< zu wählen; ein Hardcover geht natürlich auch, ist allerdings etwas teurer. Beide Bindungsmöglichkeiten haben ein professionelles Aussehen und sind sehr langlebig. Falls zum Binden ein Plastikbinderücken verwendet werden sollte -- bitte auch einen Binderücken wählen, der zur Papierstärke passt. Das sieht sonst lächerlich aus. Die Plastikbindung ist eine günstige Lösung; im Gegensatz zu anderen Bindungsmöglichkeiten wirkt es allerdings weniger professionell. Denkt schon vor dem Drucken ggf. an eine Bindungskorrektur (diese kann in der Vorlage \emph{praeambel.sty} mittels \verb|\bcor| eingestellt werden).
\end{itemize}
