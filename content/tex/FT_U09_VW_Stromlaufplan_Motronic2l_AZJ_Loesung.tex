%ju 28-Mai-22 FT_U09_VW_Stromlaufplan_Motronic2l_AZJ_Loesung.tex
\section{Aufgabenstellung Messübungen Motronic 2,0l
AZJ}\label{aufgabenstellung-messuebungen-motronic-20l-azj}

\textbf{(Ü9) Aufgabenstellung und VW Stromlaufplan S. 1-3 Erklärung}

$\to$ vom Ziel zurück zur Batterie/Generator lesen

$\to$ Elektrische Ladungen gleichen sich grundsätzlich immer dort aus,
wo sie getrennt worden sind!

$\to$ Ein elektrischer Strom kann grundsätzlich immer nur in einem
geschlossenen Stromkreis fließen.

\textbf{Aufgabe 1}

Ja, die Aussage ist zutreffend. Strompfad 101-107

\begin{itemize}
\item
  S223 Sicherung im Strompfad 99, rot, 10 A
\item
  und S163 Sicherung Strompfad 8, rot 50 A
\end{itemize}

\emph{Farben der Sicherungen} Vgl. Tabellenbuch
(\textcite{bell:2021:tabellenbuchKfz} S. 281)

\textbf{Aufgabe 2}

S243 Sicherung Strompfad 156, über die Trennstelle T4f/1 Strompfad 112
erfolgt die positive Spannungsversorgung.

\textbf{Aufgabe 3}

Kl. 3/86 + auf Kl. 5/85 - messen, Bordnetzspannung beträgt 13,8 V

Die Spannung, die am Steuerstrom Anschluss gemessen werden kann, beträgt
4,6 V.

\emph{Alternative:} die Gesamtspannung durch die Anzahl der Widerstände
dividieren. Hier in diesem Fall haben wir es mit drei gleich große
Widerstände zu tun.

3x gleich große Widerstände a $72~\Omega \quad$
$U_{Teil} = \frac{13,8~V}{3} = 4,6~V$

\textbf{Aufgabe 4}

Ja, sie werden durch zwei verschiedene Sicherungen abgesichert.

\begin{itemize}
\item
  Komponente N156 S10 mit 15 A, Strompfad 19
\item
  Komponente N261 S234 mit 10 A Strompfad 157
\end{itemize}

\textbf{Aufgabe 5}

Es wird gegen PIN1 und PIN4 jeweils ein positiver Spannungswert von 13,8
V gemessen (Bordnetzspannung), es fließt in der jetzigen Situation kein
Strom, demzufolge kann auch an der Sicherung keine Spannung abfallen.

Das \emph{Spannungsmessgerät} wird mit dem \emph{roten Clip} an PIN1
kabelbaumseitig von G6, der \emph{schwarze Clip} an PIN4 und PIN1 von G6
angeschlossen, es wird dann ein positiver Spannungswert angezeigt.

\textbf{Aufgabe 6}

Die Zündspule mit Leistungsendstufen im Strompfad 30 - 42 werden über
die Trennstelle T6d/1 mit negativem Potenzial versorgt. Die Verbindung
mit Masse erfolgt im Strompfad 180.

Masseverbindung ist im Strompfad 180 auf Masse geschadet/gelegt.

\textbf{Aufgabe 7}

\lstset{language=Python}% C, TeX, Bash, Python 
\begin{lstlisting}[
	%caption={}, label={code:}%% anpassen
][language=Python]
# geg:
U_B = 13,8 V
U_CE = 1,5 V
R_Ventil = 17 Ohm
# ges: P_E-Ventil_1.Zyl
# Formel:
I = U_Ventil / R_Ventil -> I = (U_B - U_CE) / R_Ventil
P = U x I -> P = (U_B - U_CE) x I
# Lösung:
I = 0,7235 A
P = 8,8994 W
\end{lstlisting}

\textbf{Aufgabe 8}

\lstset{language=Python}% C, TeX, Bash, Python 
\begin{lstlisting}[
	%caption={}, label={code:}%% anpassen
][language=Python]
# geg:
U_B = 13,8 V
U_CE = 1,5 V
R_Ventil = 17 Ohm
A = 0,35 mm^2 aus Stromlaufplan Nr. 77/9
l = 1,2 m
p = 0,0178 Ohm x mm^2/m
# ges: Spannungsminderung U_v_minusseitigeVersorgung_4.Zyl
# Formel:
U_Ventil = U_B - U_CE - U_v
R_l = p x l / A
R_ges = R_Ventil + R_l
I = U_k / R_ges -> I = (U_B - U_CE) / R_ges
U_v = I x R_l
# Alternative
U_v = I x p x l / A
# Lösung:
R_l = 0,061 Ohm
R_ges = 17,061 Ohm
I = 0,7209 A
U_v = 0,044 V
\end{lstlisting}

\textbf{Aufgabe 9}

\begin{enumerate}
\item
  Steuerstrom Kraftstoffpumpenrelais J17 (grün)

  \begin{itemize}
  \item
    Stromlaufplan S. 2,3,11,12,13,2
  \end{itemize}
\item
  Steuerstrom Relais J299 für Sekundärluftpumpe (rot)

  \begin{itemize}
  \item
    Stromlaufplan S. 2, \ldots{} , 11,12,13,5
  \end{itemize}
\item
  Laststrom Relais J299 für Sekundärluftpumpe, Laststrom V101 (blau)

  \begin{itemize}
  \item
    Stromlaufplan S. 2,12
  \end{itemize}
\end{enumerate}

\textbf{Aufgabe 10}

V101 - Motor für Sekundärluftpumpe

Das Bauteil ist durch die Sicherung S162 im Sicherungshalter/Batterie
(50 A) abgesichert. Strompfad 6 Stromlaufplan S. 12.
