%ju 28-Mai-22 FM_U05_Innenwiderstand_Mathebuch_Loesung.tex
\section{Innenwiderstand Übung 5
Mathebuch}\label{innenwiderstand-uebung-5-mathebuch}

Vgl. Mathebuch (\textcite{elbl:2016:technMa} S. 121)

\textbf{Aufgabe 1}

geg:

$U_q = 12,6~V$

$R_i = 0,012~\Omega = 12~m\Omega$

$R_a = 0,051~\Omega$ (Außenwiderstand)

ges: $I, U_k$

Formel:

$I = \frac{U_q}{R_{ges}} \to I = \frac{U_q}{(R_i + R_a)}$

$U_k = I \cdot R_a$

Lösung:

$I = 200,0 A$

$U_k = 10,2 V$

\textbf{Aufgabe 2}

geg:

$I = 400~A$

$U_q = 12,5~V$

$U_{k_1} = 8,3~V$ (nach 30 s), $U_{k_2} = 6,2~V$ (nach 150 s)

ges: $R_{i_1}, R_{i_2}$

Formel:

$R_{i_1} = \frac{U_{i_1}}{I} \to R_{i_1} = \frac{(U_q - U_{k_1})}{I}$

$R_{i_2} = \frac{U_{i_2}}{I} \to R_{i_2} = \frac{(U_q - U_{k_2})}{I}$

Lösung:

$R_{i_1} = 0,0105~\Omega, R_{i_2} = 0,0158~\Omega$
