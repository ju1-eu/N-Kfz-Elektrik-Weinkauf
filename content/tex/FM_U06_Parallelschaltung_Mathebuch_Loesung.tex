%ju 28-Mai-22 FM_U06_Parallelschaltung_Mathebuch_Loesung.tex
\section{Parallelschaltung Übung 6
Mathebuch}\label{parallelschaltung-uebung-6-mathebuch}

Vgl. Mathebuch (\textcite{elbl:2016:technMa} S. 82)

\textbf{Aufgabe 4a)}

geg:

$R_1 = 30~\Omega, R_2 = 20~\Omega$

$U = 12,6~V$

ges: $I_{ges}, I_1, I_2, R_{ges}$

Formel:

$I_1 = \frac{U}{R_1}, I_2 = \frac{U}{R_2}$

$I_{ges} = I_1 + I_2$

$R_{ges} = \frac{U}{I_{ges}}$

Lösung:

$I_1 = 0,42~A, I_2 = 0,63~A$

$I_{ges} = 1,05~A$

$R_{ges} = 12,0~\Omega$

\textbf{Aufgabe 4b)}

geg:

$I_1 = 0,3~A, I_2 = 0,2~A$

$U = 14,4~V$

ges: $I_{ges}, R_{ges}, R_1, R_2$

Formel:

$I_{ges} = I_1 + I_2$

$R_{ges} = \frac{U}{I_{ges}}$

$R_1 = \frac{U}{I_1}, R_2 = \frac{U}{I_2}$

Lösung:

$I_{ges} = 0,5~A$

$R_{ges} = 28,8~\Omega$

$R_1 = 48,0~\Omega, R_2 = 72,0~\Omega$

\newpage

\textbf{Aufgabe 4c)}

geg:

$I_{ges} = 1,2~A$

$I_2 = 0,3~A$

$R_{ges} = 187,5~\Omega$

ges: $I_1, R_1, R_2, U$

Formel:

$I_{ges} = I_1 + I_2 \quad \to I_1 = I_{ges} - I_2$

$U = R_{ges} \cdot I_{ges}$

$R_1 = \frac{U}{I_1}, R_2 = \frac{U}{I_2}$

Lösung:

$I_1 = 0,9~A$

$U = 225,0~V$

$R_1 = 250,0~\Omega, R_2 = 750,0~\Omega$
