% ju 12-5-22 Lerntools.tex
\documentclass[a4paper,12pt,fleqn,parskip=half]{scrartcl}
\usepackage[ngerman]{babel}
\usepackage[utf8]{inputenc}
\usepackage[T1]{fontenc}

%\usepackage{lmodern}
\usepackage[osf,sc]{mathpazo} 
\usepackage[scale=.9,semibold]{sourcecodepro}   
\usepackage[osf]{sourcesanspro}   

\usepackage[left=2cm,right=1cm,top=1cm,bottom=1cm,includeheadfoot]{geometry}
%\usepackage[left=4cm,right=2cm,top=1cm, bottom=1cm,includeheadfoot]{geometry}
%\usepackage[landscape=true,left=2cm,right=2cm,top=1cm,bottom=1cm,includeheadfoot]{geometry}%quer
\usepackage[headsepline]{scrlayer-scrpage}
\pagestyle{scrheadings}
\clearpairofpagestyles

\usepackage{xcolor}
\usepackage{amsmath}
\usepackage{amsfonts}
\usepackage{amssymb}
\usepackage{graphicx}
\usepackage{booktabs}
%\usepackage{blindtext}
\usepackage{listings}
\usepackage{eurosym}%\euro 20,-
\usepackage{url}% Links
\usepackage[inline]{enumitem}
\usepackage{pifont}

%%%%%%%%%%%%%%%%%%%%%%%%%%%%%%%%%%%%%%%%%%%%%%%%%%%%%%%
\newcommand{\name}{Jan Unger}% anpassen!!!!!
\newcommand{\thema}{Lerntools}% anpassen!!!!!
\newcommand{\quelle}{\name}
%%%%%%%%%%%%%%%%%%%%%%%%%%%%%%%%%%%%%%%%%%%%%%%%%%%%%%%

\ihead{\textbf{Quelle:} \quelle}%{Kopfzeile innen}
\ohead{\textbf{Datum:} \today}  %{Kopfzeile außen}
\ifoot{\textbf{Thema:} \thema}  %{Fußzeile  innen}
\ofoot{\pagemark}               %{Fußzeile  außen}

\title{\thema}
\author{\name}
\date{\today}
\begin{document}
%\thispagestyle{empty}
%\maketitle
%\newpage
%\setcounter{page}{1}

%%%%%%%%%%%%%%%%%%%%%%%%%%%%%%%%%%%%%%%%%%%%%

	\begin{center}
		\textbf{\Large \thema}\\%14pt
		\vspace{0.6em}
		%\datum
	\end{center}

%%%%%%%%%%%%%%%%%%%%%%%%%%%%%%%%%%%%%%%%%%%

\subsection*{Kompetenz = Wissen + Können}%\label{sec:Deadline}\index{Deadline}

% Checkliste
\begin{itemize} 
	\item [$\square$] Verhältnis: Wissen abrufen \& wiederholen (60:40) z. B.  2h $\to$ 1:15h (A) und 45 Min (W)
	\item [$\square$] ABC-Liste - Assoziatives Denken
	\begin{itemize} 
		\item Regeln: Kategorisieren (Was ist gleich?), mit den Augen wandern, mehr oder leer, Listen nummerieren und sammeln (20-25), mit Datum
		\item Vergleichen (Wissensaustausch), Fragen beantworten, Googeln $\to$ Erweitern des eigenen Wissensnetzes
	\end{itemize}
	\item [$\square$] KaWa - Wortbild (DIN-A4/A3) - Zusammenhang $\to$ >>Wissen verknüpfen, nicht isolieren!<<
	\item [$\square$] Mäntylä-Liste $\to$ Assoziieren zu Begriffen, später  Rekonstruieren der Begriffe
	\begin{itemize} 
		\item Fokus auf das Wesentliche $\to$ Abfragen nach Stunden / Woche / Monat
	\end{itemize}
	\item [$\square$] Fragen mit Lückentext für Antwort generieren
	\item [$\square$] Fragen formulieren vor Lerneinheit $\to$ Geist öffnen, Aufmerksamkeit, Interesse wecken
	\begin{itemize} 
		\item Fragen ist der Schlüssel zum Begreifen / Merken
	\end{itemize}
	\item [$\square$] Lernkarten $\to$ Begriff erklären oder umschreiben um zu erraten, Kategorisieren
	\item [$\square$] Hörtexte (Rhetoriktraining)  $\to$ Aktiv \& passiv Hören $\to$ unbewusst Lernen (Inzidentales Lernen, nebenbei)
	\begin{itemize} 
		\item [\textcircled{1}] Vorlesen und Audio- oder Videoaufzeichnung
		\item [\textcircled{2}] vertiefendes Lesen (1. so schnell wie möglich + 2. halblaut)
		\item [\textcircled{3}] Mental - Training (lautlos Lesen)
		\item [\textcircled{4}] Vorlesen und Audio- oder Videoaufzeichnung (Unterschiede zur nächsten Übungsphase wahrnehmen)
	\end{itemize}
	\item [$\square$] Stegreifrede (Rhetoriktraining) $\to$ eine Minute zum Thema reden
	\item [$\square$] Beamer - Vortrag (Rhetoriktraining)
	\begin{itemize} 
		\item Bilder: ein Bild sagt mehr als 1.000 Worte (Eisberg!)
		\item Nicht zu viele Slides (3 -- 5 Minuten pro Slide = 12 Slides für 60 Minuten Vortrag)
		\item Schrift: Überschrift Calibri Bold 36, Untertitel 28, Textkörper 24
		\item Quelle: Feodora - Adobe Stock, >>Zitat<< -- Albert Einstein
		\item Konzept: Ziel $\to$ Warum?, Inhalt $\to$ Was?, Zielgruppe $\to$ Für wen?
		\item Einleitung -- Kern -- Schluss
		\item 1-15-25x Training je mit Audio- oder Videoaufzeichnung
	\end{itemize}
\end{itemize}

\end{document}
